%==============================================================================
% Formatting parameters.
%==============================================================================

\documentclass[11pt]{article} 
\makeatletter                    % Make '@' accessible.
\pagestyle{headings}                % We do our own page headers.
\pagenumbering{arabic}

\oddsidemargin=0in                % Left margin minus 1 inch.
\evensidemargin=0in                % Same for even-numbered pages.
\textwidth=6.5in                % Text width (8.5in - margins).
\topmargin=0in                    % Top margin minus 1 inch.
\headsep=0.2in                    % Distance from header to body.
\textheight=8in                    % Body height (incl. footnotes)
\skip\footins=4ex                % Space above first footnote.
\hbadness=10000                    % No "underfull hbox" messages.
\makeatother                    % Make '@' special again.

\usepackage{epic,eepic}
\usepackage{url}
\usepackage{graphics}
\usepackage{graphicx}
\usepackage{amsmath, amssymb}
\usepackage{latexsym}
\usepackage{fullpage}
\usepackage{rotating}
\usepackage{multirow}
\usepackage{pdflscape}

%==============================================================================
% Macros.
%==============================================================================
\newcommand{\new}[1]{{\em #1\/}}        % New term (set in italics).
\newcommand{\set}[1]{\{#1\}}            % Set (as in \set{1,2,3})
\newcommand{\incrange}[1]{[#1]}         % Inclusive range
\newcommand{\setof}[2]{\{\,{#1}|~{#2}\,\}}    % Set (as in \setof{x}{x > 0})
\newcommand{\C}{\mathbb{C}}                    % Complex numbers.
\newcommand{\N}{\mathbb{N}}                     % Positive integers.
\newcommand{\Q}{\mathbb{Q}}                     % Rationals.
\newcommand{\R}{\mathbb{R}}                     % Reals.
\newcommand{\Z}{\mathbb{Z}}                     % Integers.
\newcommand{\compl}[1]{\overline{#1}}        % Complement of ...            

\begin{document}

\title{Data Mining Project Proposal}
\author{Kevin Tierney  (kevt@itu.dk), }

\maketitle

\section{Project Question}

Do stock technicals really matter?

\section{Data Sources}

We will use publicly available data in Yahoo Finance to get historical stock
prices and volumes on a day to day basis for the past several years. We can
combine this data with data from ScotTrade, which is behind a paywall, but for
which Kevin has an account.  We can calculate the technical indicators we wish
to investigate based on the stock price and volume.

\section{Algorithms}

\subsection{Supervised Learning}

Using a feature vector of technical indicators and prices we will learn when to
buy and sell a stock. The labels for data points will be either ``buy'',
``sell'', or ``hold'' which we can determine by reading the stock prices and
determining which periods of time would make money and which would lose money
if we held the stock. We may try to have ``strong buy'' and ``strong sell'' as
labels in order to have more human useful information come out of the learning.

\subsection{ Clustering }

?

\section{Time Plan}

\end{document}

