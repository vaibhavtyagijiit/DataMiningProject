%==============================================================================
% Formatting parameters.
%==============================================================================

\documentclass[11pt]{article} 
\makeatletter                    % Make '@' accessible.
\pagestyle{headings}                % We do our own page headers.
\pagenumbering{arabic}

\oddsidemargin=0in                % Left margin minus 1 inch.
\evensidemargin=0in                % Same for even-numbered pages.
\textwidth=6.5in                % Text width (8.5in - margins).
\topmargin=0in                    % Top margin minus 1 inch.
\headsep=0.2in                    % Distance from header to body.
\textheight=8in                    % Body height (incl. footnotes)
\skip\footins=4ex                % Space above first footnote.
\hbadness=10000                    % No "underfull hbox" messages.
\makeatother                    % Make '@' special again.

\usepackage{epic,eepic}
\usepackage{url}
\usepackage{graphics}
\usepackage{graphicx}
\usepackage{amsmath, amssymb}
\usepackage{latexsym}
\usepackage{fullpage}
\usepackage{rotating}
\usepackage{multirow}
\usepackage{pdflscape}

%==============================================================================
% Macros.
%==============================================================================
\newcommand{\new}[1]{{\em #1\/}}        % New term (set in italics).
\newcommand{\set}[1]{\{#1\}}            % Set (as in \set{1,2,3})
\newcommand{\incrange}[1]{[#1]}         % Inclusive range
\newcommand{\setof}[2]{\{\,{#1}|~{#2}\,\}}    % Set (as in \setof{x}{x > 0})
\newcommand{\C}{\mathbb{C}}                    % Complex numbers.
\newcommand{\N}{\mathbb{N}}                     % Positive integers.
\newcommand{\Q}{\mathbb{Q}}                     % Rationals.
\newcommand{\R}{\mathbb{R}}                     % Reals.
\newcommand{\Z}{\mathbb{Z}}                     % Integers.
\newcommand{\compl}[1]{\overline{#1}}        % Complement of ...            

\begin{document}

\title{Data Mining Project Proposal}
\author{Konstantin Kutzkov (konk@itu.dk) \\ Kevin Tierney  (kevt@itu.dk)}

\maketitle

\vspace{-40pt}

\subsection*{Project Question}

Does past stock performance really matter?

\vspace{-10pt}
\subsection*{Data Sources}

We will use publicly available data in Yahoo Finance to get historical stock
prices and volumes on a day to day basis for the past several years. We can
combine this data with data from ScotTrade, which is behind a paywall, but for
which Kevin has an account.  We can calculate the technical indicators we wish
to investigate based on the stock price and volume.

\vspace{-10pt}
\subsection*{Data Mining}

\vspace{-5pt}
\subsubsection*{Supervised Learning}

Using a feature vector of technical indicators and prices we will learn when to
buy and sell a stock. The labels for data points will be either ``buy'',
``sell'', or ``hold'' which we can determine by reading the stock prices and
determining which periods of time would make money and which would lose money
if we held the stock. We may try to have ``strong buy'' and ``strong sell'' as
labels in order to have more human useful information come out of the learning.

We plan to try support vector machines along with neural networks for this
stage of the research. We will experiment with different kernels for the SVM
and different hidden layer sizes for the neural network.

\vspace{-5pt}
\subsubsection*{Frequent Pattern Mining }

We propose to attempt to determine rules about stock movements by extending the
approach presented in \cite{fpstock}. In \cite{fpstock} each day is assigned a
category based on the price movement of the stock. These categories can then be
mined to determine rules. We will experiment with adding more categories based and more features of a stock price behavior, e.g. volume of sells, stochastic oscillator for a given period of time.

% Frequent patterns were mined
% and rules were determined, such as "Yin-Yin" is followed by "Yin" with 50\%
% confidence. We hope to include more sophisticated elements into the label
% determination, such as stock volume and technicals such as the stochastic
% oscillator and simple moving average that provide more long term context to
% stock movements.
 
We plan to use a  publicly available state-of-the sequential mining solver, e.g. Prefix Span \cite{prefixspan}. However, the exact representation as event sequences is yet to be decided.

\vspace{-10pt}
\subsection*{Time Plan}
\begin{itemize}
    \setlength{\itemsep}{1pt}
    \setlength{\parskip}{0pt}
    \setlength{\parsep}{0pt}
    \item Mid April: Prepare full dataset
    \item End April: Supervised Learning results
    \item Early May: Frequent pattern mining results
    \item May 18: Hand in report
\end{itemize}

\begin{thebibliography}{1}
\bibitem{prefixspan}
Jian Pei, Jiawei Han, Behzad Mortazavi-Asl, Jianyong Wang, Helen Pinto, Qiming Chen, Umeshwar Dayal, Meichun Hsu. 
\newblock Mining Sequential Patterns by Pattern-Growth: The PrefixSpan Approach. 
\newblock {\em IEEE Trans. Knowl. Data Eng. 16(11)}: 1424--1440 (2004)
\bibitem{fpstock}
Jo Ting, Tak-Chung Fu, Fu-Lai Chung: 
\newblock Mining of Stock Data: Intra- and Inter-Stock Pattern Associative Classification. \newblock {\em DMIN 2006:} 30--36
\end{thebibliography}
\end{document}

